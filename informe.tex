% Created 2020-06-20 Sat 22:13
% Intended LaTeX compiler: pdflatex
\documentclass[11pt]{article}
\usepackage[utf8]{inputenc}
\usepackage[T1]{fontenc}
\usepackage{graphicx}
\usepackage{grffile}
\usepackage{longtable}
\usepackage{wrapfig}
\usepackage{rotating}
\usepackage[normalem]{ulem}
\usepackage{amsmath}
\usepackage{textcomp}
\usepackage{amssymb}
\usepackage{capt-of}
\usepackage{hyperref}
\author{Jesus Rodolfo Izurieta Veliz}
\date{\today}
\title{Inform sesión de programación 2. Node Js}
\hypersetup{
 pdfauthor={Jesus Rodolfo Izurieta Veliz},
 pdftitle={Inform sesión de programación 2. Node Js},
 pdfkeywords={},
 pdfsubject={},
 pdfcreator={Emacs 26.3 (Org mode 9.4)}, 
 pdflang={English}}
\begin{document}

\maketitle
\tableofcontents


\section{Introducción}
\label{sec:org745e1cd}
Desarrollo de un servidor web que atienda peticiones de archivos con diferentes formatos (ASCII, JSON, HTML, PDF).
Los archivos estarán distribuidos en dos únicos directorios: \emph{pub y /doc).
Cuando el servidor reciba una petición GET con la ruta ‘/doc}', deberá recuperarlo desde el directorio /doc;
el mismo procedimiento deberá seguirse con la ruta ‘/pub'. Por ejemplo: /pub/test.html

El servidor deberá verificar si el archivo solicitado existe antes de recuperarlo,
de lo contrario deberá retornar una página comunicando esta condición de error (Codigo 404 – Not Found).
Use los módulos “http”, “url” y “fs” de Node.

\section{Pruebas}
\label{sec:org495b983}
Pruebas: Use un navegador para probar su servidor incluyendo distintos formatos y directorios.
\end{document}
